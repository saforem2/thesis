\makeatletter
\def\input@path{{../}}
\makeatother
\documentclass[../main.tex]{subfiles}


\begin{document}
This thesis is primarily composed of the work completed during my graduate career and includes some relevant background
information that may be useful for those looking to learn more information about how ideas from machine learning can be
applied to problems in lattice gauge theory and lattice quantum chromodynamics (QCD).
%
In particular, Chapter~\ref{chap:machine_learning} provides a thorough background on many of the machine learning tools
used throughout the remainder of the thesis and provides concrete examples on their use.
%
For example, topics such as such as supervised learning, gradient descent / backpropagation, feed-forward and
convolutional neural networks are covered.

Chapter~\ref{chap:unsupervised_learning} covers an example of how unsupervised learning (specifically, principal
component analysis) can be applied to extract information about the phase transition of the two-dimensional Ising
model.
%
By representing equilibrium configurations of the system as two-dimensional greyscale images, principal component
analysis allows us to obtain a direct relationship between the specific heat capacity and the eigenvalue of the
dominant principal component.
%
In Sec.~\ref{sec:trg} and Sec.~\ref{sec:rgimages}, a renormalization group transformation is proposed that can be
applied to generic sets of images, which when applied to the images under consideration, leads to a finite-size scaling
analysis of the critical point.

Chapter~\ref{chap:semi_supervised_learning} describes a new technique for applying supervised learning to help improve
the efficiency of Hamiltonian / Hybrid Monte Carlo (HMC) simulations.
%
This new approach is called `Learning to Hamiltonian Monte Carlo' (L2HMC), and is based off of the work described
in~\cite{2017arXiv171109268L}.
%
In Sec.~\ref{sec:l2hmc_mcmc}, a brief overview of Markov Chain Monte Carlo (MCMC) methods in general is discussed, and
the relevant notation used for the remainder of the chapter is introduced.
%
Sec.~\ref{sec:l2hmc_hmc} discusses some of the current problems faced with HMC, particularly within the context of
simulations in lattice gauge theory and lattice QCD.
%
The details of the L2HMC algorithm are described in Sec~\ref{sec:l2hmc_l2hmc}, and an example of this algorithm applied
to a two-dimensional Gaussian Mixture Model is included in Sec.~\ref{sec:l2hmc_gmm}.
%
Building on these results, we proceed to look at applying this approach to a two-dimensional $U(1)$ lattice gauge
theory, the details of which are laid-out in Sec.~\ref{sec:l2hmc_u1}.
%
Finally, in Sec.~\ref{subsec:l2hmc_modified_network}-~\ref{subsec:l2hmc_modifiedloss} we discuss some
of the modifications that were introduced when applying this approach to the gauge model under consideration, and
provide insight into why these modifications were both necessary and advantageous.
\end{document}
