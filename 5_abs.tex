\begin{doublespace}
\begin{tightcenter}
ABSTRACT
\mylinespacing%
\end{tightcenter}

% Prior to your first thesis deposit, replace this text with the text of your scientific/ scholarly abstract. The text
% of this abstract should be double spaced and each new paragraph should be indented. This text may be altered between
% first and final deposits.

%
% This thesis discusses new approaches to traditional problems in lattice gauge theory
% developments in machine learning.
In recent years there has been a growing interest in applying ideas from machine learning to computational problems in
lattice gauge theory and lattice QCD.
%


% L2HMC ABSTRACT {{{
% \begin{abstract}
%     We describe a new technique for performing Hamiltonian Monte-Carlo (HMC)
%     simulations using an alternative leapfrog integrator that is parameterized
%     by weights in a neural network.
%     %
%     We look at applying this technique to a
%     two-dimensional Gaussian Mixture Model and a two-dimensional $U (1) $
%     lattice gauge theory, and compare our results against traditional HMC\@.
%     Ongoing issues and potential areas for improvement are discussed,
%     particularly within the context of HPC and long-term goals of the lattice
%     QCD community.
%
%     % This technique is applied to a
%     % two-dimensional Gaussian Mixture Model and a two-dimensional $U(1)$
%     % lattice gauge theory, where results are compared against traditional HMC.
%     % This algorithm is called L2HMC (for `Learning To Hamiltonian Monte
%     % Carlo`)~\cite{2017arXiv171109268L}.
%
%     % demonstrated for the case of the two-dimensional Gaussian Mixture Model,
%     % This technique is then applied to the 2D $U(1)$ lattice gauge theory model.
%     % Results are then compared to those obtained from generic HMC.
%     % This algorithm is called L2HMC (for `Learning To Hamiltonian Monte
%     % Carlo`)~\cite{2017arXiv171109268L}.
% \end{abstract}
% }}}

% ISING_WORMS ABSTRACT {{{
% Using the example of configurations generated with the worm algorithm for the
% two-dimensional Ising model, we propose  renormalization group (RG)
% transformations, inspired by the tensor RG, that can be applied to sets of images. We relate criticality to
% the logarithmic divergence of the largest principal component.  We discuss the changes
% in link occupation  under the RG transformation, suggest ways to obtain data
% collapse, and compare with the two state tensor RG approximation near the fixed
% point.
% }}}
%

% Benefits and common issues encountered when using this approach are also discussed, in
% Common issues encountered when using this algorithm are also discussed before introducing an alternative
% approach called Hamiltonian (Hybrid) Monte Carlo (HMC).
% Hamiltonian (Hybrid) Monte Carlo
% (HMC) methods and describe some of the current issues it faces for certain
% types of problems, and how we believe some of these these difficulties can be
% overcome using a new technique called L2HMC ('learning to Hamiltonian Monte
% Carlo').

\mylinespacing%
\mylinespacing%
\begin{tightcenter}
\textbf{This abstract is required for everyone except DMA and MFA students.}
\end{tightcenter}
\end{doublespace}
