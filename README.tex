\hypertarget{l2hmc-qcd}{%
\subsection{l2hmc-qcd}\label{l2hmc-qcd}}

Application of the L2HMC algorithm to simulations in lattice QCD. A
description of the L2HMC algorithm can be found in the paper:

\href{https://arxiv.org/abs/1711.09268}{\emph{Generalizing Hamiltonian
Monte Carlo with Neural Network}}

by \href{http://ai.stanford.edu/~danilevy}{Daniel Levy},
\href{http://matthewdhoffman.com/}{Matt D. Hoffman} and
\href{sohldickstein.com}{Jascha Sohl-Dickstein}

\begin{center}\rule{0.5\linewidth}{\linethickness}\end{center}

\hypertarget{overview}{%
\subsubsection{Overview}\label{overview}}

\textbf{NOTE}: There are compatibility issues with
\texttt{tensorflow.\_\_version\_\_\ \textgreater{}\ 1.12} To be sure
everything runs correctly, make sure \texttt{tensorflow==1.12.x} is
installed.

\vspace{10pt}
\noindent
Given an analytically described distribution\footnote{simple examples can be
found in \texttt{l2hmc-qcd/utils/distributions.py}}, L2HMC enables training of
fast-mixing samplers.
%
\hypertarget{modified-implementation-for-lattice-gauge-theory-lattice-qcd-models.}{%
  \subsubsection[Modified Implementation]{Modified Implementation for Lattice Gauge Theory / Lattice QCD
Models}%
% \addcontentsline{toc}{subsubsection}{Modified Implementation}
\label{modified-implementation-for-lattice-gauge-theory-lattice-qcd-models.}}

This work is based on the original implementation which can be found at
\href{https://github.com/brain-research/l2hmc}{brain-research/l2hmc/}.
 
\vspace{10pt}
\noindent
My current focus is on applying this algorithm to simulations in lattice
gauge theory and lattice QCD, in hopes of obtaining greater efficiency
compared to generic HMC.

\vspace{10pt}
\noindent
This new implementation includes the algorithm as applied to the $2D
 U{(1)}$ lattice gauge theory model (i.e.~compact QED).
%
\vspace{10pt}
\noindent
Additionally, this implementation includes a convolutional neural
network architecture that is prepended to the network described in the
original paper. The purpose of this additional structure is to better
incorporate information about the geometry of the lattice.

\vspace{10pt}
\noindent
Lattice code can be found in \texttt{l2hmc-qcd/lattice/} and the
particular code for the $2D$ $U{(1)}$ lattice gauge model can be
found in \texttt{l2hmc-qcd/lattice/lattice.py}.

\hypertarget{features}{%
\subsubsection{Features}\label{features}}
%
This model can be trained using distributed training through
\href{https://github.com/horovod/horovod}{\texttt{horovod}}, by passing
the \texttt{-\/-horovod} flag as a command line argument.

\hypertarget{organization}{%
\subsubsection{Organization}\label{organization}}

% Example command line arguments can be found in \texttt{l2hmc-qcd/args}.

To run \texttt{l2hmc-qcd/gauge\_model\_main.py} using one of the
\texttt{.txt} files found in \texttt{l2hmc-qcd/args}, simply pass the
\texttt{*.txt} file as the only command line argument prepended with
\texttt{@}.
%
For example, from within the \texttt{l2hmc-qcd/args} directory:

% \begin{verbatim}
\texttt{python3 ../gauge\_model\_main.py @gauge\_model\_args.txt}
% \end{verbatim}

\vspace{10pt}
\noindent
All of the relevant command line options are well documented and can be
found in:\footnote{sample values for these arguments can be found in
\texttt{l2hmc-qcd/args/gauge\_model\_args.txt}}

\texttt{l2hmc-qcd/utils/parse\_args.py}

\vspace{5pt}
\noindent
Or by running \texttt{python3 gauge\_model\_main.py --help}.
%
% Almost all relevant information about different parameters and run options can
% be found in this file.

\vspace{10pt}
\noindent
Model information can be found in
\texttt{l2hmc-qcd/models/gauge\_model.py} which is responsible for
building the graph and creating all the relevant tensorflow operations
for training and running the L2HMC sampler.

\vspace{10pt}
\noindent
The code responsible for actually implementing the L2HMC algorithm is
dividied up between \texttt{l2hmc-qcd/dynamics/gauge\_dynamics.py} and
\texttt{l2hmc-qcd/network/}.

\vspace{10pt}
\noindent
The code responsible for performing the augmented leapfrog algorithm is
implemented in the \texttt{GaugeDynamics} class defined in
\texttt{l2hmc-qcd/dynamics/gauge\_dynamics.py}.

\vspace{10pt}
\noindent
There are multiple different neural network architectures defined in
\texttt{l2hmc-qcd/network/} and different architectures can be specified
as command line arguments defined in:

\texttt{l2hmc-qcd/utils/parse\_args.py}.

\vspace{10pt}
\noindent
\texttt{l2hmc-qcd/notebooks/} contains a random collection of jupyter
notebooks that each serve different purposes and should be somewhat self
explanatory.

\hypertarget{contact}{%
\subsubsection{Contact}\label{contact}}

\textbf{\emph{Code author:}} Sam Foreman

\noindent
\textbf{\emph{Pull requests and issues should be directed to:}}
\href{http://github.com/saforem2}{saforem2}

\hypertarget{citation}{%
\subsubsection{Citation}\label{citation}}

If you use this code, please cite the original paper:

\begin{verbatim}
@article{levy2017generalizing,
  title={Generalizing Hamiltonian Monte Carlo with Neural Networks},
  author={Levy, Daniel and Hoffman, Matthew D. and Sohl-Dickstein, Jascha},
  journal={arXiv preprint arXiv:1711.09268},
  year={2017}
}
\end{verbatim}
